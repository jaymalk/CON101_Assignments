 \documentclass{article}
    \usepackage{amsmath}
    \usepackage[utf8]{inputenc}
    \usepackage[english]{babel}
    \usepackage{color}
 
    \setlength{\parindent}{4em}
    \setlength{\parskip}{1em}
\renewcommand{\baselinestretch}{1.5}
    \usepackage{graphicx}
    
    \title{Assignment 4}
    \author{Rajbir Malik \\ 2017CS10416}
    
    \begin{document}
    
    \maketitle
    
    \begin{center}
    \Large{\underline{\textbf{Euler Forward and Backward Fitting}}}
    \end{center}
    \subsection*{Overview}
    In this assignment, we were asked to use \textbf{\emph{Euler Fittings}} for approximating the characteristics of \textit{Damped Harmonic Oscillator}. With the help ideas discussed in the class, such as \textbf{Backward Fitting} and \textbf{Forward Fitting}, I was able to plot and see the relation between error and the time gap implemented.\\I used \emph{Python} programming language for coding the assignment and the \texttt{matplotlib} module for plotting.
    
    \pagebreak
    \subsection*{Code Details}
    
    \begin{itemize}
        \item \textbf{Imports}
            \begin{itemize}
                \item random.randint
                \item math.log2
            \end{itemize}
        \item \textbf{Functions}
        \begin{center}
                \begin{center}
                    \(P(hockey \mid word) = \dfrac{P(word \mid hockey)*P(hockey)}{P(word)}\)
                \end{center}
                And, now we can define, the terms on R.H.S as follows.
                \vspace{+5mm}
                \begin{itemize}
                    \item \(P(word\mid hockey)\) = \(\dfrac{\texttt{hockey[word]}+\textcolor{red}{\textbf{1}}}{(\sum_{}^{}\texttt{hockey[keys]})+\textcolor{red}{\textbf{2}}}\) \\[2mm]
                    {\scriptsize \emph{     here, we are using 0 as the default value and red colored values for \textbf{smoothing} \\}.}
                    \item \(P(hockey) = \dfrac{N(\text{hockey in training set})}{N(\text{total lines in training set})}\) \ \\[5mm] 
                    \item \(P(word)\) = \(\dfrac{\texttt{hockey[word]}+\texttt{baseball[word]}+\textcolor{red}{\textbf{1}}}{(\sum_{}^{}\texttt{hockey[keys]})+(\sum_{}^{}\texttt{baseball[keys]})+\textcolor{red}{\textbf{2}}}\)
                \end{itemize}
                \  \\[5mm]Having all the values now, all we have to do if calculate \\
                \begin{center}
                    \(chances(hockey) = \sum_{line}(\texttt{log}(P(hockey\mid word_i))\)
                \end{center}
                And, if \(chances(hockey)\ \mathcal{>} \  chances(baseball)\), then our guess becomes \textbf{hockey}, otherwise \textbf{baseball}.
        \end{center}
    \end{itemize}
    
    \pagebreak
    \subsection*{Code Details}
    
    \begin{itemize}
        \item \textbf{Imports}
            \begin{itemize}
                \item random.randint
                \item math.log2
            \end{itemize}
        \item \textbf{Functions}
        \begin{center}
                \begin{center}
                    \(P(hockey \mid word) = \dfrac{P(word \mid hockey)*P(hockey)}{P(word)}\)
                \end{center}
                And, now we can define, the terms on R.H.S as follows.
                \vspace{+5mm}
                \begin{itemize}
                    \item \(P(word\mid hockey)\) = \(\dfrac{\texttt{hockey[word]}+\textcolor{red}{\textbf{1}}}{(\sum_{}^{}\texttt{hockey[keys]})+\textcolor{red}{\textbf{2}}}\) \\[2mm]
                    {\scriptsize \emph{     here, we are using 0 as the default value and red colored values for \textbf{smoothing} \\}.}
                    \item \(P(hockey) = \dfrac{N(\text{hockey in training set})}{N(\text{total lines in training set})}\) \ \\[5mm] 
                    \item \(P(word)\) = \(\dfrac{\texttt{hockey[word]}+\texttt{baseball[word]}+\textcolor{red}{\textbf{1}}}{(\sum_{}^{}\texttt{hockey[keys]})+(\sum_{}^{}\texttt{baseball[keys]})+\textcolor{red}{\textbf{2}}}\)
                \end{itemize}
                \  \\[5mm]Having all the values now, all we have to do if calculate \\
                \begin{center}
                    \(chances(hockey) = \sum_{line}(\texttt{log}(P(hockey\mid word_i))\)
                \end{center}
                And, if \(chances(hockey)\ \mathcal{>} \  chances(baseball)\), then our guess becomes \textbf{hockey}, otherwise \textbf{baseball}.
        \end{center}
    \end{itemize}
    
    \subsection*{Summary}
    Using the above methods, I was able to get decent accuracy. On the provided data, my accuracy ranged from 97\%-99\%, but it was never 100\%. This was a bit disappointing. But, overall, this assignment was an amazing experience. Regards. Thanks a lot!

\end{document}